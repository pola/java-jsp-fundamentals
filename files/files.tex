%\documentclass[handout]{beamer}
\documentclass{beamer}

\usepackage{listings}
\usepackage{tabularx}

\definecolor{codegreen}{rgb}{0,0.6,0}
\definecolor{codegray}{rgb}{0.5,0.5,0.5}
\definecolor{codepurple}{rgb}{0.58,0,0.82}
\definecolor{backcolour}{rgb}{0.95,0.95,0.92}

\lstdefinestyle{customjava}
{
	keywordstyle=\bfseries\color{green!40!black},
	commentstyle=\itshape\color{purple!40!black},
	identifierstyle=\color{blue},
	stringstyle=\color{orange},
	basicstyle=\ttfamily\footnotesize,
	language=java,
	breakatwhitespace=false,
	breaklines=true,
	captionpos=b,
	keepspaces=true,
	numbers=left,
	numbersep=5pt,
	showspaces=false,
	showstringspaces=false,
	showtabs=false,
	tabsize=2,
	frame=l
}

\usetheme{Copenhagen}
\setbeamertemplate{footline}{}
\setbeamertemplate{navigation symbols}{}

\title{Working with the file system}
\author{Martin Pola}
\date{}

\setlength{\parskip}{0.5cm}

\begin{document}
	\maketitle
	
	\frame
	{
		\frametitle{Persistent storage}

		The file system is \emph{persistent}, which means that what you store there will be kept until removal.

		The memory, where objects and primitives live during program execution, is volatile to program interrupts and power outages.

		\pause
		Files are stored in directories (folders).
		
		At the very basic level, all files are just sequences of bytes. To store text, one would typically apply some character encoding (e.g. UTF-8, ASCII or latin1).
	}

	\begin{frame}
		\frametitle{File management in programming}

		\emph{Saving} is known as \emph{writing} in software development.

		You cannot, however, just read or write -- consider the following.

		\begin{itemize}
			\item Files can have permissions (read, read+write) that apply to various users
			\item Java programs run with the permissions from a user account
			\pause
			\item Files can be very large -- too large for memory
		\end{itemize}

		Typical steps:

		\begin{enumerate}
			\item Open a \emph{file handler} or \emph{file descriptor} resource
			\item Set the pointer by seeking to a place in the file
			\item Perform reading or writing
			\item Close the resource
		\end{enumerate}
	\end{frame}

	\begin{frame}
		\frametitle{Reading from a file}
		\lstinputlisting[style=customjava]{RAF.read.part.java}

		Reading will advance the seek pointer by the number of read bytes.
	\end{frame}

	\begin{frame}
		\frametitle{Writing to file}
		\lstinputlisting[style=customjava]{RAF.write.part.java}

		Writing will advance the seek pointer by the number of written bytes.

		It's possible to add content to the end of the file by seeking to the end (\texttt{r.seek(r.length())}) before writing.
	\end{frame}

	\begin{frame}[fragile]
		\frametitle{Pathfinding}
	
		File paths can be \emph{relative} or \emph{absolute}. Relative paths start at the \emph{working directory}. If the working directory changes, the relative path might point to the wrong location.

		\pause

		\begin{itemize}
			\item \texttt{../../Downloads/MyFile.java} -- relative path
			\item \texttt{/Users/mpol/Downloads/MyFile.java} -- absolute path
			\item \texttt{C:\textbackslash Users\textbackslash mpol\textbackslash Downloads\textbackslash MyFile.java} -- absolute path
		\end{itemize}

		Note that you'd need to escape your backslashes on Windows!

		\begin{lstlisting}[style=customjava]
		String myPath = "C:\\Users\\mpol\\...";\end{lstlisting}
	\end{frame}

	\begin{frame}[fragile]
		\frametitle{Working directory}
		Used as a baseline for relative paths.
		
		Typically where you stood in your terminal when the program was executed. Change your working directory with \texttt{cd} in your terminal.
		
		\begin{itemize}
			\item \texttt{echo \%cd\%} in Windows command line
			\item \texttt{pwd} in Unix-like systems (e.g. OS X)
		\end{itemize}

		Use
\begin{lstlisting}[style=customjava]
System.getProperty("user.dir")\end{lstlisting}
\vspace{-0.5cm}
to see the current working directory.
	\end{frame}
\end{document}