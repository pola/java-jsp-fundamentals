%\documentclass[handout]{beamer}
\documentclass{beamer}

\usepackage{listings}

\definecolor{codegreen}{rgb}{0,0.6,0}
\definecolor{codegray}{rgb}{0.5,0.5,0.5}
\definecolor{codepurple}{rgb}{0.58,0,0.82}
\definecolor{backcolour}{rgb}{0.95,0.95,0.92}

\lstdefinestyle{customjsp}
{
	keywordstyle=\bfseries\color{green!40!black},
	commentstyle=\itshape\color{purple!40!black},
	identifierstyle=\color{blue},
	stringstyle=\color{orange},
	basicstyle=\ttfamily\footnotesize,
	language=html,
	breakatwhitespace=false,
	breaklines=true,
	captionpos=b,
	keepspaces=true,
	numbers=left,
	numbersep=5pt,
	showspaces=false,
	showstringspaces=false,
	showtabs=false,
	tabsize=2,
	frame=l
}

\usetheme{Copenhagen}
\setbeamertemplate{footline}{}
\setbeamertemplate{navigation symbols}{}

\title{Introduction to Java Server Pages (JSP)}
\author{Martin Pola}
\date{}

\setlength{\parskip}{0.5cm}

\begin{document}
	\maketitle
	
	\frame
	{
		\frametitle{Why use JSP?}

		Creating dynamic websites -- let users not only fetch resources, but also interact with them.

		There are many possibilities.

		\begin{itemize}
			\item Article comments sections
			\item Web browser-based chat applications
			\item Forums
			\item E-commerce
			\item Surveys
		\end{itemize}
	}
	
	\frame
	{
		\frametitle{Web server with JSP support}

		It's difficult to develop a web server -- at least for production use.

		\begin{itemize}
			\item Performance -- handle multiple requests simultaneously
			\item Security -- don't expose the underlying system
			\item Maintainability -- the code easily gets quite complex
		\end{itemize}

		\pause

		Instead, we'll use \emph{Tomcat}, a widely spread, free web server with JSP support.
	}
	
	\frame
	{
		\frametitle{A basic JSP file}

		Write normal HTML, but inject Java code whenever suitable. Save with the \texttt{.jsp} extension in the file name, instead of \texttt{.html}.

		\lstinputlisting[style=customjsp]{hello.part.jsp}
	}
\end{document}