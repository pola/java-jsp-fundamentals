%\documentclass[handout]{beamer}
\documentclass{beamer}

\usepackage{listings}

\definecolor{codegreen}{rgb}{0,0.6,0}
\definecolor{codegray}{rgb}{0.5,0.5,0.5}
\definecolor{codepurple}{rgb}{0.58,0,0.82}
\definecolor{backcolour}{rgb}{0.95,0.95,0.92}

\lstdefinestyle{customjava}
{
	keywordstyle=\bfseries\color{green!40!black},
	commentstyle=\itshape\color{purple!40!black},
	identifierstyle=\color{blue},
	stringstyle=\color{orange},
	basicstyle=\ttfamily\footnotesize,
	language=java,
	breakatwhitespace=false,
	breaklines=true,
	captionpos=b,
	keepspaces=true,
	numbers=left,
	numbersep=5pt,
	showspaces=false,
	showstringspaces=false,
	showtabs=false,
	tabsize=2,
	frame=l
}

\usetheme{Copenhagen}
\setbeamertemplate{footline}{}
\setbeamertemplate{navigation symbols}{}

\title{Generating documentation for Java programs}
\author{Martin Pola}
\date{}

\setlength{\parskip}{0.5cm}

\begin{document}
	\maketitle
	
	\frame
	{
		\frametitle{When is documentation necessary?}

		\begin{itemize}
			\item The code base becomes big
			\item To be able to look back at what you did several months or years ago
			\item Remembering non-trivial classes and relations between objects
			\pause
			\item \emph{When other people are supposed to interact with your code}
		\end{itemize}
	}

	\begin{frame}
		\frametitle{Javadoc}
	
		\begin{itemize}
			\item Automates the process of documentation
			\item Creates consistent HTML documentation that can be viewed in any modern browser
			\item Using Javadoc, developers who are familar with Java will know how to read the documentation
		\end{itemize}
	\end{frame}

	\begin{frame}[fragile]
		\frametitle{Preparing for Javadoc}

		Add special comments with two asterisks (*) right before class, method, enum or property declarations.

		\begin{lstlisting}[style=customjava]
/** Some description of what's being declared below */\end{lstlisting}

		\pause
		It's possible to specify metadata in the documentation.

		\begin{itemize}
			\item \texttt{@author Martin Pola}
			\item \texttt{@return An array of previously saved book objects.}
			\item \texttt{@param title The title of the book.}
		\end{itemize}

		\pause
		Navigate to the folder where the documentation should be saved and run the command:
		\begin{lstlisting}
$ javadoc C:\path\to\MyClass.java\end{lstlisting}
	\end{frame}
\end{document}