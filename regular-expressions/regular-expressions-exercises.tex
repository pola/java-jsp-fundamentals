\documentclass[a4paper, english]{article}

\usepackage[utf8]{inputenc}
\usepackage[parfill]{parskip}
\usepackage[english]{babel}
\usepackage{listings}
\usepackage{hyperref}
\usepackage{tabularx}
\usepackage{makecell}
\usepackage{xcolor}
\usepackage{calc}

\title{Regular expressions}
\author{Martin Pola}
\date{}

\definecolor{codegreen}{rgb}{0,0.6,0}
\definecolor{codegray}{rgb}{0.5,0.5,0.5}
\definecolor{codepurple}{rgb}{0.58,0,0.82}
\definecolor{backcolour}{rgb}{0.95,0.95,0.92}

\lstdefinestyle{customjava}
{
	keywordstyle=\bfseries\color{green!40!black},
	commentstyle=\itshape\color{purple!40!black},
	identifierstyle=\color{blue},
	stringstyle=\color{orange},
	basicstyle=\ttfamily\footnotesize,
	language=java,
	breakatwhitespace=false,
	breaklines=true,
	captionpos=b,
	keepspaces=true,
	numbers=left,
	numbersep=5pt,
	showspaces=false,
	showstringspaces=false,
	showtabs=false,
	tabsize=2,
	frame=l
}

\hypersetup{
    colorlinks=true,
    linkcolor=blue,
    filecolor=blue,      
    urlcolor=blue
}

\lstset{
    basicstyle=\ttfamily,
    columns=fullflexible,
}

\begin{document}
    \maketitle

    \section{Checking for valid strings}
        Come up with regular expressions for the use cases below. Test them against at least the given test strings, but try to come up with some test cases on your own, and see if they also work with your regular expressions.

        To experiment with regular expressions, \href{https://regex101.com/}{regex101.com} is a good playground.

        \vspace{0.5cm}
        \hspace{-2cm}
        \begin{tabularx}{\dimexpr(\textwidth+4cm)}{Xll}
            \textbf{Description} & \textbf{Should match} & \textbf{Should fail} \\ \hline
            Car registration plates & \makecell[l]{ABC 123 \\ ABC 73K} & \makecell[l]{ABC123 \\ abc 123 \\ ABCD 1234} \\ \hline
            CSS color codes & \makecell[l]{\#aaaaaa \\ \#C849A8 \\ \#847290 \\ red \\ green} & \makecell[l]{482cd3 \\ \#813na9 \\ \#8472906 \\ \#123xyz} \\ \hline
            E-mail addresses & \makecell[l]{martin@pola.org \\ martin.pola@academy.se \\ martin@canvas.academy.se} & \makecell[l]{martin..pola@netlight.com \\ .martin@academy.se \\ martin@example} \\ \hline
            \hline
        \end{tabularx}
    
    \section{From HTML to plain text}
        Define a String in a Java class with some HTML code, such as in this example: \\
        \texttt{<p>Hello everyone! This is some text, \\
        and here is an image.</p><img src="cat.jpg" alt="Cat" />}

        Use the \texttt{replaceAll} method in the \texttt{String} class to remove all HTML tags, using a regular expression, leaving you with only the plain text. This is the expected result: \\
        \emph{Hello everyone! This is some text, and here is an image.}

        Make sure that the regular expression works for more complex HTML structures, such as this one: \\
        \texttt{<p>Hello everyone! All animals are equal, but \\
        <span style="font-style: italic;">some animals</span> are\\
        more equal than others. \\
        <a href="https://en.wikipedia.org/wiki/Animal\_Farm">Read more \\
        at Wikipedia!</a></p>}

        The plain text version of the HTML snippet above should be: \\
        \emph{Hello everyone! All animals are equal, but some animals are more equal than others. Read more at Wikipedia!}
    
    \section{Extracting data}
        Download \href{https://archive.kottnet.net/disorganized/neuromancer.txt}{neuromancer.txt} and save it on your computer. In a Java program, load the contents of the file to a String, and then use regular expressions to extract data from the text file.

        When extracting data, the part that you would like to save must be defined as a \emph{group}. That is accomplished by surrounding it by parenthesis. An expression that previously was used to match words with only capital letters, such as \texttt{[A-ZÅÄÖ]+}, can be used as a group by changing it to: \\
        \texttt{([A-ZÅÄÖ]+)}

        Can you, for instance, find any double names? Here is an outline of what the code could look like -- make sure to have parenthesis in your regular expression to capture groups:

        \lstinputlisting[style=customjava]{RegexFinder.part.java}

        Ideaas of things can you can look for:

        \begin{itemize}
            \item Double names (e.g. \emph{Anna-Klara})
            \item Words with an initial capital letter followed by at least ten lowercase letters (e.g. \emph{Intercontinental})
            \item Words that contain the letter \texttt{s} at least four times (e.g. \emph{consciousness})
        \end{itemize}

        \paragraph{Stretch task} Given results from the findings above, can you see which word that is the most common? Can you build a highscore of words, perhaps by using a \texttt{HashMap} with the word as the key and the frequency count as value?
\end{document}