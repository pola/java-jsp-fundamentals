%\documentclass[handout]{beamer}
\documentclass{beamer}

\usepackage{listings}
\usepackage{tabularx}

\definecolor{codegreen}{rgb}{0,0.6,0}
\definecolor{codegray}{rgb}{0.5,0.5,0.5}
\definecolor{codepurple}{rgb}{0.58,0,0.82}
\definecolor{backcolour}{rgb}{0.95,0.95,0.92}

\lstdefinestyle{customjava}
{
	keywordstyle=\bfseries\color{green!40!black},
	commentstyle=\itshape\color{purple!40!black},
	identifierstyle=\color{blue},
	stringstyle=\color{orange},
	basicstyle=\ttfamily\footnotesize,
	language=java,
	breakatwhitespace=false,
	breaklines=true,
	captionpos=b,
	keepspaces=true,
	numbers=left,
	numbersep=5pt,
	showspaces=false,
	showstringspaces=false,
	showtabs=false,
	tabsize=2,
	frame=l
}

\usetheme{Copenhagen}
\setbeamertemplate{footline}{}
\setbeamertemplate{navigation symbols}{}

\title{Regular expressions}
\author{Martin Pola}
\date{}

\setlength{\parskip}{0.5cm}

\begin{document}
	\maketitle
	
	\frame
	{
		\frametitle{Matching strings}

		A regular expression is a combination of a \emph{pattern} and a \emph{test string}.
		
		\pause
		The pattern is sent to a \emph{regular expression engine} to see if the test string matches or not.

		\pause
		Two purposes.
		\begin{itemize}
			\item Testing if a string is valid
			\item Extracting information from a string
		\end{itemize}
	}

	\begin{frame}
		\frametitle{Examples}

		\begin{tabularx}{\textwidth}{lX}
			\textbf{Expression} & \textbf{Description} \\ \hline
			\texttt{[0-9]} & Any single digit \\
			\texttt{[0-9A-ZÅÄÖ]+} & Sequence of digits and capital letters \\
			\texttt{[A-ZÅÄÖ][a-zåäö]+} & One capital letter followed by at least one lowercase letter \\
			\texttt{[0-9]\{4\}} & Four digits \\
			\texttt{[0-9]\{4,6\}} & Four to six digits \\
			\texttt{[0-9]\{4,\}} & At least four digits \\
			\texttt{[0-9]\{,4\}} & Up to four digits \\
			\texttt{.\{3,\}} & Anything, at least tree times \\
			\texttt{[a-zåäö]?} & Zero or one lowercase letters \\
			\texttt{[0-9]\{4\}-[0-9]\{2\}-[0-9]\{2\}} & Date-like information (YYYY-MM-DD) \\
			\texttt{(Hello|Hey|Hi|([0-9]+))} & A greeting or a number \\
		\end{tabularx}
	\end{frame}

	\begin{frame}
		\frametitle{Regular expressions in Java}

		Java provides us with the \texttt{Pattern} and \texttt{Matcher} classes, through the \texttt{java.util.regex} package.

		\lstinputlisting[style=customjava]{RegexMatcher.part.java}

		\pause
		A circumflex (\texttt{\^}) can be used to denote the start of a string, and a dollar sign (\texttt{\$}) to denote the end of a string.
	\end{frame}

	\begin{frame}[fragile]
		\frametitle{Escaping}

		Sometimes you want the literal character instead of its special meaning within regular expressions. For that, \emph{escaping} is necessary.

		Escaping is done with a backslash (\texttt{\textbackslash}), just like it's done when dealing with special characters in strings (e.g. \texttt{\textbackslash n}, \texttt{\textbackslash t}).

		\pause
		Consider this regular expression to match a Swedish phone number: \\
		\texttt{(\textbackslash+46|0046|0)[1-9][0-9]\{8\}}

		\pause
		In Java, it would correspond to:
\begin{lstlisting}[style=customjava]
Pattern.compile("(\\+46|0046|0)[1-9][0-9]{8}");\end{lstlisting}

		\pause
		Other characters to escape: \\
		\texttt{. ? - [ ] \^ \$ | ( )}
	\end{frame}
\end{document}