%\documentclass[handout]{beamer}
\documentclass{beamer}

\usepackage{listings}

\definecolor{codegreen}{rgb}{0,0.6,0}
\definecolor{codegray}{rgb}{0.5,0.5,0.5}
\definecolor{codepurple}{rgb}{0.58,0,0.82}
\definecolor{backcolour}{rgb}{0.95,0.95,0.92}

\lstdefinestyle{customjsp}
{
	keywordstyle=\bfseries\color{green!40!black},
	commentstyle=\itshape\color{purple!40!black},
	identifierstyle=\color{blue},
	stringstyle=\color{orange},
	basicstyle=\ttfamily\footnotesize,
	language=html,
	breakatwhitespace=false,
	breaklines=true,
	captionpos=b,
	keepspaces=true,
	numbers=left,
	numbersep=5pt,
	showspaces=false,
	showstringspaces=false,
	showtabs=false,
	tabsize=2,
	frame=l
}

\usetheme{Copenhagen}
\setbeamertemplate{footline}{}
\setbeamertemplate{navigation symbols}{}

\title{Forms and cookies in JSP}
\author{Martin Pola}
\date{}

\setlength{\parskip}{0.5cm}

\begin{document}
	\maketitle
	
	\frame
	{
		\frametitle{HTML frontend}

		A form can be created in pure HTML, with a range of pre-defined types of input fields.

		\begin{itemize}
			\item Line input
			\item Multiple line input (\texttt{textarea})
			\item Radio buttons
			\item Checkbox
			\item Dropdown (\texttt{select})
		\end{itemize}

		\pause

		In order to handle the data once submitted, one would usually have some server-side logic. That cannot be pure HTML.
	}
	
	\frame
	{
		\frametitle{A sample form}

		\lstinputlisting[style=customjsp]{form.part.jsp}

		\pause

		\begin{itemize}
			\item The \texttt{action} attribute in the \texttt{form} tag specifies where the browser should go upon submission
			\item The \texttt{name} attribute in the \texttt{input} tag can be used to obtain the value at the server (with JSP)
		\end{itemize}
	}
	
	\frame
	{
		\frametitle{Handle form submission}

		Look at the URL in the browser when the form has been submitted: \\
		\texttt{...:8080/form-handler.jsp?fullname=Martin+Pola\&age=24}

		The part after the question mark (\texttt{?}) is called the \emph{query string}.

		\lstinputlisting[style=customjsp]{form-handler.jsp}
	}
	
	\frame
	{
		\frametitle{From stateless to stateful}

		HTTP is \emph{stateless}. Each request is handled individually and multiple requests from the same browser are not linked together.

		Sometimes, a stateful design is desirable -- e.g. to keep track of a user who is logged in.

		Cookies let us save data in the browser, and have the browser return the data with every sub-sequent request.

		\begin{itemize}
			\item First, the server sends a \texttt{Set-Cookie} header to save cookie data
			\item On every subsequent request, the web browser includes a \texttt{Cookie} header containing previously saved cookies
		\end{itemize}
	}
	
	\frame
	{
		\frametitle{Saving a cookie}

		\lstinputlisting[style=customjsp]{form-handler-save-cookie.jsp}
	}
	
	\frame
	{
		\frametitle{Reading a cookie}

		\lstinputlisting[style=customjsp]{form-handler-read-cookie.jsp}
	}
\end{document}