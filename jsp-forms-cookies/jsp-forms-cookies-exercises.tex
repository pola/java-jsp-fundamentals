\documentclass[a4paper, english]{article}

\usepackage[utf8]{inputenc}
\usepackage[parfill]{parskip}
\usepackage[english]{babel}
\usepackage{listings}
\usepackage{hyperref}
\usepackage{xcolor}

\definecolor{codegreen}{rgb}{0,0.6,0}
\definecolor{codegray}{rgb}{0.5,0.5,0.5}
\definecolor{codepurple}{rgb}{0.58,0,0.82}
\definecolor{backcolour}{rgb}{0.95,0.95,0.92}

\lstdefinestyle{customjsp}
{
	keywordstyle=\bfseries\color{green!40!black},
	commentstyle=\itshape\color{purple!40!black},
	identifierstyle=\color{blue},
	stringstyle=\color{orange},
	basicstyle=\ttfamily\footnotesize,
	language=html,
	breakatwhitespace=false,
	breaklines=true,
	captionpos=b,
	keepspaces=true,
	numbers=left,
	numbersep=5pt,
	showspaces=false,
	showstringspaces=false,
	showtabs=false,
	tabsize=2,
	frame=l
}

\title{Forms and cookies in JSP}
\author{Martin Pola}
\date{}

\hypersetup{
    colorlinks=true,
    linkcolor=blue,
    filecolor=blue,      
    urlcolor=blue
}

\lstset{
    basicstyle=\ttfamily,
    columns=fullflexible,
}

\begin{document}
    \maketitle

    \section{Creating a feedback form}
        In \texttt{webapps/ROOT}, create a file named \texttt{feedback.jsp} with a \texttt{textarea} to allow the user to enter multiple lines of text. You can copy \href{https://klistra.in/gGlqrs8v}{this sample feedback.jsp} if you wish to get some inspiration.

        Visit \texttt{http://127.0.0.1:8080/feedback.jsp} and make sure that you can see the form and enter some text.

        Create a file called \texttt{send-feedback.jsp} to handle the scenario when a user submits the form from above. In the file, first just print the submitted data:
\begin{lstlisting}[style=customjsp,language=java]
String feedback = request.getParameter("feedback");
out.write(feedback);\end{lstlisting}

        Save the file and visit the page with your form. Enter some feedback and see to that submitting the form will redirect you to the \texttt{feedback.jsp} page, which displays the entered text.

        Add code in \texttt{send-feedback.jsp} to store the feedback text in a text file on the file system instead of printing it to the web browser. Make sure to print \emph{something}, though, such as \emph{Thank you for your feedback} or similar.

        Now, enter some more feedback in hour feedback form and submit it. Open your text file and see that the text you entered is present there.

        What happens if you submit the form multiple times? Try it out and look at your text file!

        \begin{enumerate}
            \item Can you get the newly submitted feedback to the saved at the end of the text file, instead of overwriting the current content?
            \item Can you store a timestamp with each feedback submission in the text file, showing when a feedback item was submitted?
        \end{enumerate}

    \newpage
    \section{Should feedback be anonymous?}
        Add an input field for entering a single line of text to your form, and allow the user to enter their name. Make sure to give the \texttt(input) tag a proper value to the \texttt{name} attribute, so that you can use it to retrieve what the user entered in \texttt{send-feedback.jsp}.

        Edit your form handler file (\texttt{send-feedback.jsp}) to include the name of the user along with the feedback message, in the text file.

        Make sure that your form handler file also sets a cookie with the name of the user.

        Go back to \texttt{feedback.jsp}. Add code to read the cookie with the user's name. If such cookie exists, let the \texttt{input} tag for the name have the value of the cookie as its default value. Note that you would have to use JSP tags (\texttt{<\%} and \texttt{\%>}) to print the value of the cookie.
        
        That can be accomplished by setting the \texttt{value} attribute, like this:
\begin{lstlisting}[style=customjsp]
<input
   type="text"
   name="fullname"
   value="<% out.write("some value here"); %>"
/>\end{lstlisting}

        Try to enter some data in the form and submit it. If you open the form in a new tab, will the name seem to have been saved?

        What happend if you open another browser, such as Internet Explorer, Edge or Firefox? Does the form work? Is the name still set automatically?

\end{document}