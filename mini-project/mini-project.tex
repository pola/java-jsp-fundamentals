\documentclass[a4paper, english]{article}

\usepackage[utf8]{inputenc}
\usepackage[parfill]{parskip}
\usepackage[english]{babel}
\usepackage{listings}
\usepackage{hyperref}

\title{Mini project}
\author{Martin Pola}
\date{}

\hypersetup{
    colorlinks=true,
    linkcolor=blue,
    filecolor=blue,      
    urlcolor=blue
}

\lstset{
    basicstyle=\ttfamily,
    columns=fullflexible,
}

\begin{document}
    \maketitle

    \section*{Purpose and goal}
        With the mini project, you should hopefully be able to apply many of the topics we have covered during the past two weeks in a more or less complete product. Either choose one of the suggestions below or talk to your Java trainer if you have idea for a project on your own.

        The mini project is ideally carried out in groups of 2-4 people. It is strongly recommended not to work on a project alone.

        The suggestions below have several iterations. That means you should be able to first get a \emph{inimal viable product}, which you can then extend with other features. Between iterations, make sure to take a backup of your code, so you can go back to a working state should you have issues in a future iteration.

    \section*{Project suggestions}
        \subsection*{The blog platform}
            Create a page where users can read and save posts to a blog platform. Each blog post entry could have a title and a body and blog post entries could be saved in files; one file per entry. It's wise to have a dedicated folder that is used only for blog post entries. Have one JSP file for reading the blog (listing all entries), one JSP file with a form for creating entries and a third JSP file to save entries that users submit through the form.

            \subsubsection*{Date and time}
                Add timestamps to your blog post entries, in the files, and display them on the page where you can read blog posts.
            
            \subsubsection*{Password protection}
                Don't let anyone create blog posts; protect the form with a password and require users to enter the password in the form when creating blog posts entries.
                
                Additionally, add an \emph{author} to each blog post, and make it so that the author name depends on which password the user entered. An idea could be to store a mapping of password and authors as a \texttt{Map} (\texttt{HashMap}) in the JSP file that handles submission of blog post entries.
            
            \subsubsection*{Read blog posts on a separate page}
                Instead of displaying all blog post entries on the same page, only display the titles and let users click on a title to read the blog post. In a new JSP file, take the name of the blog post file a user wishes to read as part of the \emph{query string} and, on that page, display only the blog post from that file.
        
        \subsection*{The booking system}
            Allow users to book the washhouse (sv. \emph{tvättstuga}) online! Have a form where users can choose a day using a \texttt{select} tag in HTML, and optionally add a comment. Let the user select any day between today's date and the next thirty days. Save the booking in a text file with the date as the name of the text file.
            
            On another page, display a list of dates, starting today and spanning over the next thirty days. Under each date, list any bookings that are saved.
            
            \subsubsection*{Unbook}
                Allow users to remove bookings by clicking a link that leads to a JSP file. Send the date of the booking in the \emph{query string} and remove the file from disk if it exists.
            
            \subsubsection*{Multiple bookings on a single day}
                Let multiple people book the same day. First, get it working without times, and then, perhaps also add times (e.g. 18:00-21:00). An idea for data storage could be to store all data for a single day in a single file, and separate multiple bookings using a special character (e.g. a \textbackslash t -- the tab).
    
    \section*{Tips and tricks}
        \paragraph{Cookies} Regardless of which project you choose, can you find a way to utilize cookies to simplify the experience for the user? For the booking system, perhaps you can let the user have a cookie with their booked date, to highlight that booking when they view the page.

        \paragraph{Class with good-to-use methods} Some things are very commonly used, but still require multiple lines of code. Examples could be reading from a file or generating a random string. Implement those lines of code as static methods in a class that you can then call from anywhere in your program, to quickly perform those common tasks. \href{https://klistra.in/UvlLVbb5}{CommonTasks.java} is an example of that. (Compile it and place it in \texttt{webapps/ROOT/WEB-INF}!)
\end{document}