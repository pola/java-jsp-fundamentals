%\documentclass[handout]{beamer}
\documentclass{beamer}

\usepackage{listings}

\definecolor{codegreen}{rgb}{0,0.6,0}
\definecolor{codegray}{rgb}{0.5,0.5,0.5}
\definecolor{codepurple}{rgb}{0.58,0,0.82}
\definecolor{backcolour}{rgb}{0.95,0.95,0.92}

\lstdefinestyle{customjava}
{
	keywordstyle=\bfseries\color{green!40!black},
	commentstyle=\itshape\color{purple!40!black},
	identifierstyle=\color{blue},
	stringstyle=\color{orange},
	basicstyle=\ttfamily\footnotesize,
	language=java,
	breakatwhitespace=false,
	breaklines=true,
	captionpos=b,
	keepspaces=true,
	numbers=left,
	numbersep=5pt,
	showspaces=false,
	showstringspaces=false,
	showtabs=false,
	tabsize=2,
	frame=l
}

\usetheme{Copenhagen}
\setbeamertemplate{footline}{}
\setbeamertemplate{navigation symbols}{}

\title{Networking and HTTP}
\author{Martin Pola}
\date{}

\setlength{\parskip}{0.5cm}

\begin{document}
	\maketitle
	
	\frame
	{
		\frametitle{Foundations of the Internet}
		Many smaller networks that are linked together: \emph{inter}-net.

		\begin{itemize}
			\item Devices on the internet are called \emph{hosts}
			\item All hosts have at least one IP address
			\item Multiple IP addresses necessary if a host has multiple \emph{network interfaces} (e.g. both wired and wireless)
		\end{itemize}

		There are different \emph{versions} of IP addresses.

		\begin{itemize}
			\item IP version 4 -- 192.168.10.2
			\item IP version 6 -- 2001:0db8:85a3:0000:0000:8a2e:0370:7334
		\end{itemize}

		IP version 6 (IPv6) was introduced to mitigate the issue of IP address exhaustion.
	}

	\begin{frame}
		\frametitle{Find your IP address}
	
		\begin{itemize}
			\item \texttt{ipconfig} on Windows
			\item \texttt{ifconfig} on Unix-like systems (abbreviation for \emph{interface config})
		\end{itemize}

		You can always reach your own computer (or host, more generally) via the \emph{loopback interface}: \\
		\texttt{127.0.0.1}
	\end{frame}

	\begin{frame}
		\frametitle{Hosts, routers and switches}
		It would be impossible for every host to have direct connections to all other hosts.

		\next
	
		\begin{itemize}
			\item Switches are used \emph{inside} networks
			\item Routers are used \emph{between} networks
		\end{itemize}
	\end{frame}

	\begin{frame}
		\frametitle{Sockets}
		Two-way communication between (a pair of) hosts on the internet.

		\begin{itemize}
			\item Client-server -- client makes requests, server responds
			\item Peer-to-peer -- hosts are both clients and servers
		\end{itemize}
	\end{frame}

	\begin{frame}
		\frametitle{Services}
		Sockets provide a foundation for \emph{services} on the Internet.
		
		Through sockets, hosts exchange sequences of bytes. For hosts to understand each other, services use \emph{protocols} -- pre-defined, standardized languages.

		\pause
		Most services use the client-server model.

		\begin{itemize}
			\item Hypertext Transfer Protocol (HTTP) -- our \emph{web}
			\item Internet Message Access Protocol (IMAP) -- reading e-mail
			\item Simple Mail Transfer Protocol (SMTP) -- sending e-mail
			\item Skype
		\end{itemize}

		\pause
		There are also services using the peer-to-peer model, e.g. Bittorrent.
	\end{frame}

	\begin{frame}
		\frametitle{HTTP}

		\begin{enumerate}
			\item The client (your web browser) makes a request
			\item The server responds to the request
		\end{enumerate}

		Requires a host to run a program that acts like a server. A host can run multiple server programs at the same time.

		How does the host know which server program an incoming connection is trying to reach? Each server program \emph{listens} on a separate port number.
		
		HTTP runs on port 80, by default, but the port number can be specified in the URL.

		\begin{itemize}
			\item \texttt{http://example.net/} -- default port
			\item \texttt{http://example.net:8080/} -- port 8080
		\end{itemize}
	\end{frame}

	\begin{frame}[fragile]
		\frametitle{HTTP request}

		\begin{lstlisting}
GET / HTTP/1.0
Host: example.net\end{lstlisting}

		The slash (\texttt{/}) comes from the last part of the URL: \\
		\texttt{http://example.net\underline{/}}

		We can use \texttt{telnet} to connect to a remote server:
		\begin{lstlisting}
$ telnet example.net 80\end{lstlisting}
	\end{frame}

	\begin{frame}[fragile]
		\frametitle{HTTP response}

		A response consists of \emph{headers} and a \emph{body}. The two sections are separated by \underline{two} lines breaks.

		\begin{lstlisting}
HTTP/1.1 200 OK
Content-Type: text/html; charset=UTF-8
Expires: Wed, 12 Feb 2020 09:01:17 GMT
Last-Modified: Thu, 17 Oct 2019 07:18:26 GMT
Server: ECS (nyb/1D04)
Content-Length: 1256

<!doctype html>
<html>
<head>
[...]\end{lstlisting}
	\end{frame}
\end{document}