%\documentclass[handout]{beamer}
\documentclass{beamer}

\usepackage{listings}

\definecolor{codegreen}{rgb}{0,0.6,0}
\definecolor{codegray}{rgb}{0.5,0.5,0.5}
\definecolor{codepurple}{rgb}{0.58,0,0.82}
\definecolor{backcolour}{rgb}{0.95,0.95,0.92}

\lstdefinestyle{customjava}
{
	keywordstyle=\bfseries\color{green!40!black},
	commentstyle=\itshape\color{purple!40!black},
	identifierstyle=\color{blue},
	stringstyle=\color{orange},
	basicstyle=\ttfamily\footnotesize,
	language=java,
	breakatwhitespace=false,
	breaklines=true,
	captionpos=b,
	keepspaces=true,
	numbers=left,
	numbersep=5pt,
	showspaces=false,
	showstringspaces=false,
	showtabs=false,
	tabsize=2,
	frame=l
}

\usetheme{Copenhagen}
\setbeamertemplate{footline}{}
\setbeamertemplate{navigation symbols}{}

\title{Automated testing in Java}
\author{Martin Pola}
\date{}

\setlength{\parskip}{0.5cm}

\begin{document}
	\maketitle
	
	\frame
	{
		\frametitle{Practise of testing}

		\begin{itemize}
			\item Validate that a program does what is expected
			\item Run tests periodically -- catch errors early
			\item Manual tests are humans following steps in documents
			\item Automatic tests are carried out through the use of a \emph{test framework}
		\end{itemize}

		\pause
		\begin{itemize}
			\item Testing doesn't find bugs
			\item Testing only makes correctness \emph{probable}, it doesn't \emph{prove} it
		\end{itemize}
	}

	\frame
	{
		\frametitle{Testing in Java}

		\textbf{Unit tests} \\
		Test single features, typically a method call.

		\textbf{Integration tests} \\
		Test that different parts of a program can integrate with each other, e.g. a database or a network service.

		\pause
		We will use \emph{JUnit} as our test framework.

		\begin{itemize}
			\item A unit test should \emph{assert} something.
			\item \emph{Expected value} is compared with \emph{actual value}
		\end{itemize}
	}

	\frame
	{
		\frametitle{Concrete example}

		\lstinputlisting[style=customjava]{example.java}
		
		\pause
		Methods for verifying assertion are found in the \texttt{Assert} class.
		
		\begin{itemize}
			\item \texttt{assertEquals(\underline{expected}, \underline{actual})}
			\item \texttt{assertTrue(\underline{actual})}
			\item \texttt{assertFalse(\underline{actual})}
			\item \texttt{assertNotNull(\underline{actual})}
			\item ...
		\end{itemize}
	}

	\frame
	{
		\frametitle{How many tests should you create?}

		\begin{itemize}
			\item Many tests run quickly; thousands of tests in just a few seconds
			\item What if you need to change the tests?
		\end{itemize}
		
		\pause
		How many lines from our program do our tests touch?
		
		\pause
		Measure \emph{code coverage} and aim to cover all lines of code.
	}
\end{document}